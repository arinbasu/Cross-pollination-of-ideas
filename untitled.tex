\section{Testing}

This is an experiment to write using both authorea and overleaf and update each other. Can we create a git bridge using dropbox that will keep our files in sync? I am not sure we can do that but it's worth a try. Both Authorea and Overleaf are powerful writing environments and each has its own strengths that are worth exploring. 

\subsection{The Experiment}

The idea was to write in Authorea and pull in the changes to Overleaf.
Or write in Overleaf and pull in the changes to Authorea. 
Connect the two and reap benefits of each other. 

The two environments are quite different and each has its own appeal. I use Authorea for writing the paper and publishing it out as a blog post and getting the responses. I use Overleaf for final pushing of the document as a PDF document and send it out to the particular journal or creating a set of slide decks. 

The question is, can we use both gainfully? So that each remains separate on its own but say each can be updated to perform tasks that are unique to each? Or perhaps, say to edit an Authorea document, you will need to have your co-authors create an account on Authorea but they are not interested, but you can channel your contents to Overleaf from Authorea so that they still get to work on a piece of the work where they can read and edit. 

The idea is not to use the tools as either/or, but more in sync with each other. I found one way to do that is to merge the best features of both and get our work done. Both are web based, and both have their strengths that are worth exploring. Using dropbox, git, and pandoc it is possible to contribute to both systems simultaneously and update each system but different parts of both systems. As both systems have visual editors with very similar levels of visual controls, it makes sense to write in both systems using a web browser and nothing else. Besides, using markdown, pandoc and git, it is possible to update both systems in the same way using something like a jupyter notebook. 

Jupyter notebooks can also be used in Authorea to store images and analyses. This is a plus point, as everything then keeps in sync. Besides, if you write in Authorea you can draw in bibtex citations and create bibliographies on the fly and create neat tables easily that are very helpful. 

