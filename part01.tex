\section{How to use Authorea and Overleaf together}

I use both Authorea and Overleaf. Although both of these web based applications are somewhat similar, there are complementary features that make combining both of these tools very useful. Each tool has its own strengths. For instance, Authorea makes it very easy and intuitive to import articles or existing articles from MS Word, and also makes it very easy to author either in markdown or in html (in a wysiwyg). Overleaf, on the other hand, is an excellent programme to write your paper in \LaTeX and then repurpose the same paper in a number of different channels and formats. Using Overleaf, it is also very intuitive to submit to tens of journals and online services. 

Each of them have strengths that are unique to each platform. Therefore, it would be really nice to be able to play one with the other. For instance, you can develop a paper entirely in Authorea but you can also use the features of Overleaf to push it to repositories. Overleaf for instance is great for presenting PDF and also provides templates for posters, beamer presentations, and other channels. On the other hand, Authorea provides a very neat system of presenting a scholarly paper in web friendly readable format. It'd be nice to have both the systems play well with each other. On the one hand, you can gain from the superb web based presentation that Authorea offers, and on the other, you can gain from channeling a subset of the content of your paper or analysis to Overleaf which, in turn, can be used to create publishable PDFs and other formats (say posters and beamers slide decks).  

Below, I describe a method that I use to combine both Authorea and Overleaf for authoring papers. My original idea was to update Authorea from Overleaf and update Overleaf from Authorea. It turns out that you can keep the two web based services separate although it is quite possible to update one from the other. 

The way that I have connected the two are using Dropbox and Github. Both Authorea and Overleaf provide you git bridges. Overleaf on the top of it provides sync-ing with Dropbox (this is only for their paid version), but if you do not want to use Dropbox, then you can use a separate Git bridge for Overleaf. A little more work but you will gain nevertheless. 

\subsection{The Steps of Setting them up}

The idea was to write in Authorea and pull in the changes to Overleaf.
Or write in Overleaf and pull in the changes to Authorea. 
Connect the two and reap benefits of each other. 

The two environments are quite different and each has its own appeal. I use Authorea for writing the paper and publishing it out as a blog post and getting the responses. I use Overleaf for final pushing of the document as a PDF document and send it out to the particular journal or creating a set of slide decks. 

The question is, can we use both gainfully? So that each remains separate on its own but say each can be updated to perform tasks that are unique to each? Or perhaps, say to edit an Authorea document, you will need to have your co-authors create an account on Authorea but they are not interested, but you can channel your contents to Overleaf from Authorea so that they still get to work on a piece of the work where they can read and edit. 

The idea is not to use the tools as either/or, but more in sync with each other. I found one way to do that is to merge the best features of both and get our work done. Both are web based, and both have their strengths that are worth exploring. Using dropbox, git, and pandoc it is possible to contribute to both systems simultaneously and update each system but different parts of both systems. As both systems have visual editors with very similar levels of visual controls, it makes sense to write in both systems using a web browser and nothing else. Besides, using markdown, pandoc and git, it is possible to update both systems in the same way using something like a jupyter notebook. 

Jupyter notebooks can also be used in Authorea to store images and analyses. This is a plus point, as everything then keeps in sync. Besides, if you write in Authorea you can draw in bibtex citations and create bibliographies on the fly and create neat tables easily that are very helpful. 

Steps:
\begin{enumerate}
\item Create authorea paper first
\item Create github bridge
\item put the bridge in a dropbox folder
\item create an overleaf paper and save in dropbox folder
\item Interact with github with Authorea and dropbox with Overleaf (save to dropbox)
\item upload to Overleaf from github
\item in Authorea remove the old folder from overleaf if you like
\end{enumerate}


There is an advantage of working this way. On the one hand you can route any section that you may have directly to Overleaf for presentation. On the other hand, you can store part of the paper that you are working on or writing within Authorea itself. This is how you can work between the two powerful systems. 

